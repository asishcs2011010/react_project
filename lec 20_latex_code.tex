%%% Scribe template for CS5200 approximation algorithms.
%%% Adapted with some modifications from UC-Berkeley template, and Prahladh Harsha (TIFR) scribe templates


%%%%%%%%%% Do Not Alter these lines %%%%%%%%%%%%%
\documentclass[11pt]{article}
\usepackage{amsmath}
\usepackage{amssymb,amsfonts}
\usepackage{amsthm}
\usepackage{fullpage}
\usepackage{graphicx}
\usepackage[bookmarks,colorlinks,breaklinks]{hyperref}  % PDF hyperlinks, with coloured links
\hypersetup{linkcolor=blue,citecolor=blue,filecolor=blue,urlcolor=blue} % all blue links
\numberwithin{equation}{section}
\newcommand{\lref}[2][]{\hyperref[#2]{#1~\ref*{#2}}}
\renewcommand{\eqref}[2][]{\hyperref[#2]{(\ref*{#2})}}
\setlength{\oddsidemargin}{.25in}
\setlength{\evensidemargin}{.25in}
\setlength{\textwidth}{6in}
\setlength{\topmargin}{-0.4in}
\setlength{\textheight}{8.5in}
\usepackage{algorithm}
\usepackage{algorithmic} %% For writing algorithms

%%% Theorem environments %%%%%%%%

\newtheorem{theorem}{Theorem}
\newtheorem{corollary}[theorem]{Corollary}
\newtheorem{lemma}[theorem]{Lemma}
\newtheorem{observation}[theorem]{Observation}
\newtheorem{proposition}[theorem]{Proposition}
\newtheorem{claim}[theorem]{Claim}
\newtheorem{fact}[theorem]{Fact}
\newtheorem{assumption}[theorem]{Assumption}

\theoremstyle{definition}
\newtheorem{definition}[theorem]{Definition}
\newtheorem{remark}[theorem]{Remark}

\newcommand{\handout}[5]{
   \renewcommand{\thepage}{#1-\arabic{page}}
   \renewcommand{\thetheorem}{#1.\arabic{theorem}}
   \renewcommand{\thesection}{#1.\arabic{section}}
   \noindent
   \begin{center}
   \framebox{
      \vbox{
    \hbox to 5.90in { {\bf CS5200: Approximation Algorithms}
     	 \hfill #2 }
       \vspace{4mm}
       \hbox to 5.90in {  {\Large \hfill #5  \hfill} }
       \vspace{2mm}
       \hbox to 5.90in { {\it #3 \hfill #4} }
      }
   }
   \end{center}
{\bf Disclaimer}: {\it These notes have not been subjected to the
	usual scrutiny reserved for formal publications.  They may be distributed
	outside this class only with the permission of the Instructor.}
%\vspace*{2mm}
   %\vspace*{4mm}
}
\newcommand{\lecture}[5]{\handout{#1}{#2}
{Lecturer: #3}
{Scribe: #4}
{Lec #1:  #5}}

%%%%%%%%%%%%%%%%%%%%%%%%%%%%%%%%%%%%%%


%%%% Bibtex/References  stuff
\renewcommand{\cite}[1]{[#1]}
\def\beginrefs{\begin{list}%
		{[\arabic{equation}]}{\usecounter{equation}
			\setlength{\leftmargin}{2.0truecm}\setlength{\labelsep}{0.4truecm}%
			\setlength{\labelwidth}{1.6truecm}}}
	\def\endrefs{\end{list}}
\def\bibentry#1{\item[\hbox{[#1]}]}



%%%%%%%%%%%%%%%%%%%%%%%%%%%%%%%%%%

%%% Macros for approximation algorithms
\mathchardef\mhyphen="2D  %%% for hyphenation in math-mode


\newcommand{\Z}{{\mathbb Z}}
\newcommand{\R}{{\mathbb R}}
\newcommand{\Q}{{\mathbb Q}}
\newcommand{\F}{{\mathbb F}}


\newcommand{\poly}{\text{poly}}
\renewcommand{\epsilon}{\varepsilon}
\newcommand{\eps}{\epsilon}
\renewcommand{\phi}{\varphi}
\newcommand{\ALG}{\mathsf{ALG}}
\newcommand{\OPT}{\mathsf{OPT}}



%%%%%%%%%%%%%%%%%%%%%% ADD / MODIFY BELOW THIS POINT ONLY %%%%%%%%%%%%%%%%%%%%%%%%%%%%


%%%%%%%%% Scribe's Macros%%%%%%%%%%%%%
% You can add your own macros here. E.g. to define a new command that is useful for the lecture
%E.g. here define the name for "Min-Vertex-Cover" that occurs in this lecture
\newcommand{\minVC}{\mathsf{Min\mhyphen VC}}
\newcommand{\minSC}{\mathsf{Min\mhyphen Set \mhyphen Cover}}




%%%%%%%%%%%%%%%%%%%%%%%%%%%%%%%%%

\begin{document}
%%%% Fill in the approrpiate details on this line details %%%%%%%%%%%%%%%%%%
%%\lecture{lecture number}{date}{name of lecturer}{name of scribe}{title of lecture} 
%% eg: for the first lecture scribed by XYZ, it'd be
\lecture{20}{6 Apr 2023}{Rakesh Venkat}{C.Asish}{Algorithm for Minimum Vertex Cover}


%% **** YOUR NOTES GO HERE:

% Some general latex examples and examples making use of the
% macros follow.  
%**** IN GENERAL, BE BRIEF




\section{Primal Dual algorithm for shortest path}

In this section, we state Primal LP and Dual LP for shortest s-t path problem. We also visualise those Linear Programs. 

\begin{definition}[shortest s-t path problem ]
Given an undirected graph G  = (V,E), non-negative costs $c_e \geq 0$ on all edges $e \in E$, and a pair of distinguished vertices s and t. The goal is to find minimum-cost path from s to t 
\end{definition}

%%% Note how theorems are referenced, to enable hyperlinks. can reference sections etc also using $\lref$.


\subsection{Primal LP}

Consider variables  $x_{e}$ for each edge e, such that 

    $$ x_e = \begin{cases}
    1 & \text{ edge e is  selected in shortest path } \\
    0 & \text{ othewise}
    
    \end{cases}
$$

    The Primal Linear Program can be formulated as 
    
\begin{align*}
\min \sum_e c_ex_e \\
\text{subject to: }  \sum_{e \in \delta(S) }x_e\ \geq 1, \hspace{1em} \forall  S\\
x_e \in \{0,1\}, \hspace{1em}  \forall e \in E
\end{align*}

Where S be the  all the s-t cuts in the graph and 
$\delta(S)$ is the set of all edges that have one endpoint in S and the other endpoint not in S

\subsection{Dual LP}

    The Dual Linear Program can be formulated as 

\begin{align*}
\max \sum_s y_s \\
\text{subject to: }  \sum_{s : e \in \delta(S) }y_s\ \leq c_e, \hspace{1em} \forall  e \in E\\
y_s \geq 0, \hspace{1em}  \forall s \in S
\end{align*}

Where S be the  all the s-t cuts in the graph and 
$\delta(S)$ is the set of all edges that have one endpoint in S and the other endpoint not in S

\subsection{Visualisation}

Here the dual variables $y_S$ can be interpreted as "moats" surrounding the set S of width  $y_S$. and any path from s to t must cross this moat so we can say that the cost should be at least  $y_S$.

Moats must be non overlapping, and thus for any edge e, we cannot have edge e crossing moats of total width more than $c_e$; thus the dual constraint $\sum_{s : e \in \delta(S) }y_s\ \leq c_e$.

We can have many moats, and any s-t path must cross them all, and have a total length at most $ \sum_{s\in S} y_s $

\subsection{Algorithm}

\begin{algorithm}
	\caption{Primal Dual Algorithm}
	\label{alg:GreedyVC}
 
	\textbf{Input:} Graph $G=(V,E)$ \\
	\textbf{Output:} $S \subseteq V$ that  forms a vertex cover


	\begin{algorithmic}[1] 
		\STATE $y \leftarrow \emptyset$ \\ $F \leftarrow \emptyset$
		\WHILE { no s-t path in (V,F)} 
		\STATE let C be the connected component of (V,F) Containing s,  increase $y_C$ until there is an edge $e^{`} \in \delta(C)$ such that $\sum_{s\in S : e^{`} \in \delta(S) }y_s\ = c_e^{`}$.\\
        \STATE $ F \leftarrow F \cup {e^{`}}$
        \ENDWHILE
        \RETURN s-t path in (V,F)
        
	\end{algorithmic}
\end{algorithm}

	\begin{claim}
		$F$ is always a tree 
	\end{claim}
	\begin{proof}[Proof Of Claim]
         In every step of loop we add an edge $e^{`}$ from the set $\delta$(C) to our solution F,where C is a connected component of (V,F) containing s.and as exactly one endpoint of $e^{`}$ is in C, $e^{`}$ cannot close a cycle in the tree F(i.e the other end of the edge $e^{`}$ will not be in F ) and  therefore a cycle cannot be formed 
	\end{proof}

	\begin{claim}
	  Algorithm 1.1.4 returns a shortest s-t path 
	\end{claim}
	\begin{proof}[Proof Of Claim]
      Let P be the path returned by the algorithm  \\
      For every edge $e\in P $, we know $c_e =  \sum_{s: e\in \delta(S) }y_s $  \\
      and  $\sum_{e \in P }c_e$ =  $\sum_{e \in P }\sum_{s: e\in \delta(S) }y_s $  = $\sum_{S: s\in S,t\notin S}|P \cap \delta (S) |y_s$\\
       if we can show that whenever $y_s>0$, we have $|P\cap \delta (S)| = 1$ \\ then we will show that  $\sum_{e \in P }c_e  = \sum_{S: s\in S,t\notin S}y_s \leq OPT$ \\
       and we know P is a s-t path of cost no less than OPT, it must have cost exactly OPT.

       proof:- as we only increase $y_s$ when F is a spanning tree of S, and as P cannot come back into S else it would form a cycle within tree on S 
       
	\end{proof}

\section{Introduction to Steiner Forest problem}

\begin{definition}[Steiner Forest problem ]
Given an undirected graph G  = (V,E), non-negative costs $c_e \geq 0$ for all edges $e \in E$, and k pairs of  vertices $s_i$ and $t_i$. The goal is to find minimum-cost subset edges $F \subseteq E$ such that every  $s_i$ - $t_i$ pair is connected in the set set of selected edges; that is  $s_i$ a and  $t_i$ are connected in (V,F) for all i 
\end{definition}

\begin{claim}
	There is a Algorithm which gives 2-approximation for the generalized steiner tree problem 
	\end{claim}
 
\section*{References}
\beginrefs
\bibentry{CW87}{\sc David P.Williamson } and {\sc David B.Shmoys}, 
``The Design of Approximation Algorithms ''
\endrefs

%% Add in other references here, as appropriate.

\end{document}



